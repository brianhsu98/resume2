%-------------------------------------------------------------------------------
%	SECTION TITLE
%-------------------------------------------------------------------------------
\cvsection{Work Experience}


%-------------------------------------------------------------------------------
%	CONTENT
%-------------------------------------------------------------------------------
\begin{cventries}

%---------------------------------------------------------
    \cventry
      {Software Engineering Intern, Data Management Backend}
      {LiveRamp}
      {San Francisco, California}
      {May 2019 - Aug 2019}
      {
        \begin{cvitems}
          \item Worked with a variety of big data systems, helping \textbf{add to, segment, and process petabytes of customer data} to enable data-driven marketing.
          \item Developed and owned a backend service, including a \textbf{new big data pipeline}, for a new product. Worked under and met a tight deadline to meet client demands, unlocking \$12 million in at-risk revenue.
          \item \textbf{Containerized applications} using Docker and Kubernetes, increasing development velocity, enabling scalability, and improving fault tolerance.         
          \item \textbf{Optimized the performance} of mission-critical applications, and \textbf{increased visibility into errors} by adding fault-detection logic.
          \item Collaborated across teams, \textbf{implementing new endpoints to enable easier access} to my team's systems.
          \item Migrated several applications from an on-premises data center to \textbf{Google Cloud Platform} as part of a company-wide shift to the cloud.
        \end{cvitems}
      }

%---------------------------------------------------------
    \cventry
      {Research Assistant, UC Berkeley}
      {Algorithms for Computing and Education (ACE) Lab}
      {Berkeley, California}
      {May 2018 - Present}
      {
        \begin{cvitems}
          \item Worked with PhD student Nate Weinman, under Professor Armando Fox, to \textbf{research and develop novel computer science practice problems}
          to make computer science more accessible and easier-to-learn for beginning and intermediate students.
          \item Collaboratively designed and implemented an \textbf{interactive web application} allowing students to solve Parsons Problems, enabling a 80+ student study. 
          \item Developed a system for automatically \textbf{grading student submissions at scale} safely and efficiently, using multiple workers coordinated using Redis/RQ.
          \item Analyzed and visualized data, providing insights into the learning efficacy of Parsons Problems.
        \end{cvitems}  
      }

\end{cventries}
